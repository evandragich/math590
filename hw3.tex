% Options for packages loaded elsewhere
\PassOptionsToPackage{unicode}{hyperref}
\PassOptionsToPackage{hyphens}{url}
%
\documentclass[
]{article}
\title{Homework 3}
\author{Evan Dragich}
\date{22 February 2022}

\usepackage{amsmath,amssymb}
\usepackage{lmodern}
\usepackage{iftex}
\ifPDFTeX
  \usepackage[T1]{fontenc}
  \usepackage[utf8]{inputenc}
  \usepackage{textcomp} % provide euro and other symbols
\else % if luatex or xetex
  \usepackage{unicode-math}
  \defaultfontfeatures{Scale=MatchLowercase}
  \defaultfontfeatures[\rmfamily]{Ligatures=TeX,Scale=1}
\fi
% Use upquote if available, for straight quotes in verbatim environments
\IfFileExists{upquote.sty}{\usepackage{upquote}}{}
\IfFileExists{microtype.sty}{% use microtype if available
  \usepackage[]{microtype}
  \UseMicrotypeSet[protrusion]{basicmath} % disable protrusion for tt fonts
}{}
\makeatletter
\@ifundefined{KOMAClassName}{% if non-KOMA class
  \IfFileExists{parskip.sty}{%
    \usepackage{parskip}
  }{% else
    \setlength{\parindent}{0pt}
    \setlength{\parskip}{6pt plus 2pt minus 1pt}}
}{% if KOMA class
  \KOMAoptions{parskip=half}}
\makeatother
\usepackage{xcolor}
\IfFileExists{xurl.sty}{\usepackage{xurl}}{} % add URL line breaks if available
\IfFileExists{bookmark.sty}{\usepackage{bookmark}}{\usepackage{hyperref}}
\hypersetup{
  pdftitle={Homework 3},
  pdfauthor={Evan Dragich},
  hidelinks,
  pdfcreator={LaTeX via pandoc}}
\urlstyle{same} % disable monospaced font for URLs
\usepackage[margin=1in]{geometry}
\usepackage{graphicx}
\makeatletter
\def\maxwidth{\ifdim\Gin@nat@width>\linewidth\linewidth\else\Gin@nat@width\fi}
\def\maxheight{\ifdim\Gin@nat@height>\textheight\textheight\else\Gin@nat@height\fi}
\makeatother
% Scale images if necessary, so that they will not overflow the page
% margins by default, and it is still possible to overwrite the defaults
% using explicit options in \includegraphics[width, height, ...]{}
\setkeys{Gin}{width=\maxwidth,height=\maxheight,keepaspectratio}
% Set default figure placement to htbp
\makeatletter
\def\fps@figure{htbp}
\makeatother
\setlength{\emergencystretch}{3em} % prevent overfull lines
\providecommand{\tightlist}{%
  \setlength{\itemsep}{0pt}\setlength{\parskip}{0pt}}
\setcounter{secnumdepth}{-\maxdimen} % remove section numbering
\usepackage{actuarialsymbol}
\ifLuaTeX
  \usepackage{selnolig}  % disable illegal ligatures
\fi

\begin{document}
\maketitle

\hypertarget{section}{%
\subsection{2.2}\label{section}}

\[
\begin{aligned}
e_{x} &= \px{x}(1 + e_{x + 1}) \\ \\
e_{x} &= \sum_{k = 1}^{\infty} \px[k]{x} \\
&= \px{x} + \sum_{k = 2}^{\infty}\px{x} \times \px[k - 1]{k + 1} \\
&= \px{x}(1 + \sum_{k = 2}^{\infty}\px[k - 1]{k + 1}) \\
&= \px{x}(1 + e_{x + 1})
\end{aligned}
\]

Thus, by removing the first term from the sum and rewriting, we have
found an alternate form to express \(e_{x}\).

\[
\begin{aligned}
e_{60} &= \px{60}(1 + e_{61}) \\ 
\px{x} &= \frac{e_{x}}{(1 + e_{x + 1})} \\\\
\px{60} &= \frac{e_{60}}{(1 + e_{61})} = \frac{15.96}{16.27} \\
\px{61} &= \frac{e_{61}}{(1 + e_{62})} = \frac{15.27}{15.60} \\
\px{62} &= \frac{e_{62}}{(1 + e_{63})} = \frac{14.60}{14.94} \\ \\ \\
\px[3]{60} &= \px[1]{60} \times \px[1]{61} \times \px[1]{62} \\
&= \frac{15.96}{16.27} \times \frac{15.27}{15.60} \times \frac{14.60}{14.94} \\ 
&= 0.93834
\end{aligned}
\]

We can find one-year survival probability values, \(\px[1]{x}\), by
rearranging the above formula and plugging in our given \(e_{x}\)
values. Then, we can express \(\px[3]{x}\) as the product of three
sequential \(\px[1]{x}\) probabilities, arriving at
\(\px[3]{60} = 0.93834\).

\hypertarget{section-1}{%
\subsection{2.7}\label{section-1}}

\hypertarget{a.}{%
\subsubsection{a.}\label{a.}}

The survival function \(S_{0}(t)\) is defined as the complement of the
distribution function \(F_{0}t\). Thus, we can simply subtract from 1 to
arrive at \(S_{0}(t)\)

\[
\begin{aligned}
S_{0}(t) &= 1 - F_{0}t \\
&= 1 - (1 - e^{-\lambda t}) \\
&= e^{-\lambda t}
\end{aligned}
\]

\hypertarget{b.}{%
\subsubsection{b.}\label{b.}}

\[
\begin{aligned}
\mu_{x} &= \lim_{dx \rightarrow 0^{+}} \frac{F_{0}(x + dx) - F_{0} (x)}{dx(1 - F_{0}(x))} \\
&= \frac{F'_{0}(x)}{(1 - F_{0}(x))} \\
&= \frac{f_{0}(x)}{S_{0}(x)} \\
&= \frac{-S'_{0}(x)}{S_{0}(x)} \\
&= - \frac{d}{dx} ln(S_{0}(x))\\
&= - \frac{d}{dt} ln(e^{-\lambda t}) \\
&= - \frac{d}{dt} (- \lambda t) \\
&= \lambda
\end{aligned}
\]

We can derive \(\mu_{x}\), the force of mortality, as the negative
natural logarithm of the survival function, which is \(\lambda\) in this
case.

\hypertarget{b.-1}{%
\subsubsection{b.}\label{b.-1}}

\[
\begin{aligned}
e_{x} &= \sum_{k = 1}^{\infty} \px[k]{t} \\
&= \sum_{k = 1}^{\infty} S_{x}(k) \\
&= \sum_{k = 1}^{\infty} e^{-\lambda k} \\
&= \frac{e^{-\lambda}}{1 - e^{-\lambda}} \\
&= \frac{1}{e^{\lambda}- 1} 
\end{aligned}
\]

We can treat this expression as an infinite sum, and evaluate as such.
The first term, when \(k = 1\), is \(e^{-\lambda}\), with a common ratio
of \(e^{-\lambda}\). For this sum to converge,
\(\mid e^{-\lambda k}\mid < 1\), which is achieved when \(\lambda > 0\).
Then, we can evaluate and arrive at
\(e_{x} = \frac{1}{e^{\lambda}- 1}\).

\hypertarget{d.}{%
\subsubsection{d.}\label{d.}}

No, I do not believe this is reasonable to model mortality. The
associated survival function initially sharply decreases, before
gradually levelling off as age approaches infinity. This is due to the
constant force of mortality \(\mu_{x} = \lambda\).

However, this is contrary to the observed phenomenon of human mortality,
where death does not begin to pick up at a significant rate until later
in life. Thus, we would want to choose a survival function that more
appropriately models the force of mortality as increasing over age, and
thus would include \(x\) in the expression for \(\mu_{x}\) such as the
Gompertz or Makeham.

\hypertarget{section-2}{%
\subsection{2.9}\label{section-2}}

\[
\begin{aligned}
\frac{d}{dx} \px[t]{x} &= \px[t]{x} (\mu_{x} - \mu_{x + t}) \\ \\
\frac{d}{dx} \px[t]{x} &=  \frac{d}{dt}(S_{x}(t)) \\
&= \frac{d}{dt}exp(- \int_{0}^{t} \mu_{x + s}ds) \\
&= -(\mu_{x + t} - \mu_{x}) \times exp(- \int_{0}^{t} \mu_{x + s}ds) \\
&= (\mu_{x} - \mu_{x + t} ) \times exp(- \int_{0}^{t} \mu_{x + s}ds) \\
&= (\mu_{x} - \mu_{x + t} ) \times S_{x}(t) \\
&= \px[t]{x} (\mu_{x} - \mu_{x + t} )
\end{aligned}
\]

If we evaluate the derivative using the chain rule, we see that the
derivative of \(\px[t]{x}\) at a certain \(x\) is just \(\px[t]{x}\)
scaled by a factor of \((\mu_{x} - \mu_{x + t} )\). Writing these
concepts as such provides more insight on the role of the force of
mortality, \(\mu\), as the negated, unit-scaled, instantaneous rate of
change in the survival function \(S_{0}\) at a particular \(x\).

\hypertarget{section-3}{%
\subsection{2.18}\label{section-3}}

\hypertarget{a.-1}{%
\subsubsection{a.}\label{a.-1}}

\[
\begin{aligned}
\px[t]{x}^{*}  &=(\px[t]{x})^{2} \\ \\
\px[t]{x}^{*}  &= exp(- \int_{0}^{t} \mu_{x + s}^{*}ds) \\
&= exp(- \int_{0}^{t} 2 \mu_{x + s}ds) \\
&= exp(- 2 \int_{0}^{t} \mu_{x + s}ds) \\
&= \left(exp(-\int_{0}^{t} \mu_{x + s}^{}ds)\right)^2 \\
&= (\px[t]{x})^{2}
\end{aligned}
\]

We begin by substituting \(\mu_{x + s}^{*} = 2 \mu_{x + s}\) into our
expression. From there, we can extract the factor of 2 and rearrange to
where the exponentiated term is itself raised to the power of 2. From
there, the definiton of \(\mu_{x + s}\) follows and we see that the
doubling of the force of mortality translates into a squaring of the
survival function.

\end{document}
